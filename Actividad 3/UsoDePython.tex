\documentclass[a4paper]{article}
\usepackage[spanish]{babel}
\usepackage[utf8]{inputenc}
\usepackage[T1]{fontenc}
\usepackage{graphicx}
\usepackage{subfig}
\usepackage[a4paper,top=3cm,bottom=1.5cm,left=3cm,right=3cm,marginparwidth=1.75cm]{geometry}
\title{Uso de Python en el manejo de datos}
\author{Paul D. Cañez Moranda}
\begin{document}
\begin{titlepage}
\newcommand{\HRule}{\rule{\linewidth}{0.5mm}}
\center
\textsc{\LARGE Universidad de Sonora}\\[1cm]
\HRule \\[0.4cm]
{ \huge \bfseries Actividad No.3 }\\[0.4cm]
{ \huge \bfseries Manejo de datos con en Python }\\[0.4cm]
\HRule \\[1.1cm]
\begin{minipage}{0.4\textwidth}
\begin{flushleft} \large
\emph{\bfseries {Alumno:}}\\

Paul Cañez Miranda

\end{flushleft}
\end{minipage}
\begin{minipage}{0.4\textwidth}
\begin{flushright} \large
\emph{\bfseries {Profesor:}} \\
Carlos Lizzárraga Celaya 
\end{flushright}
\end{minipage}\\[0.5cm]
{\large \today}\\[1cm] 
\begin{center}
\includegraphics[width=8cm]{unison.png}
\end{center}
\end{titlepage}

\begin{abstract}
En esta práctica aprenderemos formas de manejar datos, utilzando bibliotecas de Python.
\end{abstract}

\section*{Introducción}
El uso de Emacs y los filtros de Bach, obtendremos y "limpiaremos" los datos sobre la atmósfera previamente descargados de la página de la Universidad de Wyoming.. Los datos que manejaremos serán CAPE (Convective Available Potencial Energy) y PW (Precipitable Water), obtenidos durante un año. Después con el uso de Jupyter, haremos tablas y gráficas de los datos para analizarlos de forma más práctica.

\section*{Desarrollo}
Primero, con el uso de los filtros egrep y sed, clasificaremos los datos capturados durante un año en dos grupos, uno de cada horario (00Z y 12Z) con los siguiente comandos: \begin{verbatim}
cat sondeos.txt | egrep -i "Observations|CAPE|precipitable"| sed -e "/00Z/,+2d" > 12Zanual.txt
cat sondeos.txt | egrep -i "Observations|CAPE|precipitable"| sed -e "/12Z/,+2d" > 00Zanual.txt

\end{verbatim}  
Donde cat muestra los datos del archivo "sondeos.txt". El filtro egrep sirve para indicar las variables que se desean extraer. Sed es para indicar qué grupo (horario), se quiere quitar (en el primer comando se quta las 00Z); y por último el nombre del archivo.\\

Después realizaremos tablas y gráficas para los datod CAPE Y PW.
\section*{Jupyter}
Jupyter es un programa que proporciona una gran cantidad de herramientas para la informática científica mediante paquetes interactivos que combinan la ejecución de código con la creación de documentos activos. El lenguaje utilizado por Jupyter es Python. \\ \\
Una vez que se tengan los datos ordenados, con las bibliotecas que ofrece Python tabularemos y graficaremos las variables CAPE y PW. \\ \\ 
Las bibliotecas utilisadas serán Matplotlib y Pandas. 
\begin{itemize}
\item Matplotlib\\
Esta biblioteca sirve para hacer muchos tipos de gráficas. Desde plots, gráficas de barras, histogramas, etc. 
\item Pandas\\
Las bibliotecas Pandas sirven para hacer análisis de datos. \\

\end{itemize}
\newpage

\section*{Tablas}
Con la ayuda de Pandas, tabulamos (por días) los datos de CAPE Y PW. \\
Para esto es necesario primero leer el documento en Jupyter, lo hacemos con el comando: \begin{verbatim}
df = pd.read_csv("/home/usuario/Carpeta/archivo.csv, names=['Day','CAPE','PW']")
\end{verbatim} 
donde los nombres pertenecen a las columnas. Una vez que corra, escribimos el comando \begin{verbatim}
df.head(10)
\end{verbatim}
y nos aparecerá una tabla con 10 datos como la siguiente 
\begin{verbatim}

Date	   Day 	        CAPE    	PW
0      	02 Oct 2016 	3039.83 	60.86
1      	03 Oct 2016 	5422.12 	63.33
2 	     04 Oct 2016 	3308.22 	59.03
3 	     05 Oct 2016 	3433.50 	59.91
4  	    06 Oct 2016 	4492.75 	60.57
5 	     07 Oct 2016 	2169.27 	57.27
6 	     08 Oct 2016 	3477.26 	60.64
7 	     09 Oct 2016 	3941.28 	55.10
8 	     10 Oct 2016 	3511.02 	60.29
9 	     11 Oct 2016 	2984.33 	56.00
\end{verbatim}

Esta tabla no muestra todos los datos del año, sólo muestra los primeros 10 sondeos.\\ \\ \\
En la siguiente tabla, sin embargo, se muestra datos relevantes; mínimo, máximo, media, cuartiles, etc. de todo un año de sondeos. Esta tabla la generaremos con el comando que aparece arriva de la tabla:\\
\begin{verbatim}
df.describe()

 	     CAPE 	        PW
count 	71.000000 	   71.000000
mean 	 1299.801268 	 42.679014
std 	  1354.650404 	 10.222792
min 	  0.000000 	    23.750000
25% 	  160.170000 	  36.645000
50% 	  898.960000 	  40.710000
75% 	  1987.460000 	 50.230000
max 	  5422.120000 	 63.330000
\end{verbatim}
Con esta tabla es más sencillo y útil analizar los datos.

\newpage
\section*{Gráficas}
Para realizar gráficas en Jupyter en necesario introducir el comando \begin{verbatim}
matplotlib inline
\end{verbatim}
Que es una de las bibliotecas de Python.\\ \\
En la primer gráfica vemos la cantidad de energía potencial disponible para la convectividad (CAPE) en función de la altura. El comando para obtener esta grafica es: \begin{verbatim}
df_clean['CAPE'].hist(bins=25)
\end{verbatim} 

\begin{center}
\includegraphics[width=12cm]{CAPEH.png}
\end{center}
se puede observar cómo el CAPE va disminuyendo con la altura.\\

En la siguiente gráfica se muestra cómo varia la cantidad de agua precipitable con la altura. Introducimos el siguiente comando para obtenerla: \begin{verbatim}
df_clean['PW'].hist(bins=25)
\end{verbatim}
\begin{center}
\includegraphics[width=12cm]{PWH.png}
\end{center}
podemos observar que el PW es más disperso (varía más) en comparación del CAPE.
\newpage

Otra forma de analizar de forma sencilla y eficiente un grupo de datos, es con diagramas de cajas. En ellos se puede observar el valor de los cuartiles, media, disperción, y algunas otras cosas interesantes. \\
Para graficar éstas cajas de CAPE y PW, utilizaremos los comandos: \begin{verbatim}
df_clean.boxplot(column='CAPE')
df_clean.boxplot(column='PW')
\end{verbatim}
respectivamente.\\ \\

Caja de CAPE\\ \\
\begin{center}
\includegraphics[width=12cm]{CAPECaja.png}
\end{center}
Caja de PW\\
\begin{center}
\includegraphics[width=12cm]{PWCaja.png}
\end{center}
\newpage
\section*{Sondeos durante el año 2016}

\begin{table}[htbp]
\begin{center}
\resizebox{!}{4.5cm}{
\begin{tabular}{|l|l|l|}
\hline 
Mes & 00Z & 12Z \\ \hline
Enero & 0 & 22 \\ \hline
Febrero & 0 & 18\\ \hline
Marzo & 0 & 26\\ \hline
Abril & 0 & 29 \\ \hline
Mayo & 0 & 15 \\ \hline
Junio & 0 & 26 \\ \hline
Julio & 0 & 29 \\ \hline
Agosto & 0 & 28 \\ \hline
Septiembre & 0 & 29 \\ \hline
Octubre & 21 & 30 \\ \hline
Noviembre & 27 & 30 \\ \hline
Diciembre & 0 & 0 \\ \hline
\end{tabular}}
\caption{observaciones realizadas a las 00Z y 12Z horas.}
\end{center}

\end{table}
\newpage
\section*{Conclusión}
Las bibliotecas de Python son de gran utilidad para manejar datos de forma sencilla. Además Jupyter me pareció práctico, porque puede correr todos los comandos a la vez.

\begin{thebibliography}{10}
\bibitem{m1} \textsc{$Wikipedia.org. \ Convective\ Available\ Potential\ Energy$}
\bibitem{m2} \textsc{$http://matplotlib.org/$}
\end{thebibliography}
\end{document}
