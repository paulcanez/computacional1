\documentclass[a4paper]{article}
\usepackage[spanish]{babel}
\usepackage[utf8]{inputenc}
\usepackage[T1]{fontenc}
\usepackage{graphicx}
\usepackage{subfig}
\usepackage[a4paper,top=3cm,bottom=1.5cm,left=3cm,right=3cm,marginparwidth=1.75cm]{geometry}
\title{Preguntas de la act. 2}
\author{Paul D. Cañez Moranda}

\begin{document}
\begin{titlepage}
\newcommand{\HRule}{\rule{\linewidth}{0.5mm}}
\center
\textsc{\LARGE Universidad de Sonora}\\[1cm]
\HRule \\[0.4cm]
{ \huge \bfseries Actividad No.2 }\\[0.4cm]
{ \huge \bfseries Preguntas sobre la actividad }\\[0.4cm]
\HRule \\[1.1cm]
\begin{minipage}{0.4\textwidth}
\begin{flushleft} \large
\emph{Alumno:}\\

Paul Cañez Miranda

\end{flushleft}
\end{minipage}
\begin{minipage}{0.4\textwidth}
\begin{flushright} \large
\emph{Profesor:} \\
Carlos Lizzárraga Celaya 
\end{flushright}
\end{minipage}\\[0.5cm]
{\large \today}\\[1cm] 
\begin{center}
\includegraphics[width=8cm]{unison.png}
\end{center}
\end{titlepage}
\begin{itemize}
\section*{Preguntas}
\item {\bfseries ¿Cuál es tu primera impresión del uso de bash/Emacs?} \\
Es un editor con muchas funciones que no conocía.
\item {\bfseries ¿Ya lo habías utilizado?}\\
Sí, pero sólo había visto pocas cosas.
\item {\bfseries ¿Qué cosas se te dificultaron más en bash/Emacs?}\\
Aprenderme los comandos de Emacs.
\item {\bfseries ¿Qué ventajas le ves a Emacs?}\\
Muchas, el manejo de datos se hace de forma práctica.
\item {\bfseries ¿Qué es lo que más te llamó la atención en el desarrollo de esta actividad?}\\
Aprender más funciones de Emacs. Vimos unas pocas y creo qe tiene muchas más.
\item {\bfseries ¿Qué cambiarías en esta actividad?} \\
Nada, es una actividad interesante.
\item {\bfseries ¿Qué consideras que falta en esta actividad?} \\
No creo que le falte nada.
\item {\bfseries ¿Puedes compartir alguna referencia nueva que consideras útil y no se haya contemplado?} \\
No creo, la referencia que puse en el documento es de la imagen del globo.
% Comentario
\end{itemize}
\end{document}
