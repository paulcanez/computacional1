\documentclass[a4paper]{article}

%% Language and font encodings
\usepackage[english]{babel}
\usepackage[utf8x]{inputenc}
\usepackage[T1]{fontenc}

%% Sets page size and margins
\usepackage[a4paper,top=3cm,bottom=2cm,left=3cm,right=3cm,marginparwidth=1.75cm]{geometry}


%% Useful packages
\usepackage{newcent}
\usepackage{graphicx}
\usepackage[colorinlistoftodos]{todonotes}
\usepackage[colorlinks=true, allcolors=blue]{hyperref}

\title{\Huge Atmósfera}
\author{\LARGE Paul Cañez Miranda \\ \\ \LARGE Carlos Lizárraga Celaya}
\begin{document}
\maketitle

\vspace{0.5cm}

\subsection*{\Large Introducción}

En este trabajo trataré de dar una idea de lo que es la atmósfera. Conocer cuál es su estructura, su composición física y química. También se hablará de algunos fenómenos que ocurren en esta. 

\subsection*{\Large ¿De qué está compuesta la atmósfera terrestre?}
Se conoce como atmósfera terrestre a una capa delgada, comparada con el tamaño de la tierra, que rodea al planeta. Se compone de diversos gases, en su mayoría se compone de Nitrógeno, Argón, Oxígeno y otros gases menos comunes, pero que representan un gran peligro para la vida en el planeta. \\
La estructura de la atmósfera consiste en diversas capas, cada una con sus características propias. 
Existen varias formas de clasificar o dividir la atmósfera, y esto depende de lo que se quiera estudiar. 


\subsection*{\Large Estructura de la atmósfera}

Debido a su alta comprensibilidad, la atmósfera está muy estratificada verticalmente. Aunque la delimitación de estas capas depende del fenómeno a estudiar y sus límites no son muy nítidos, suelen considerarse cuatro capas "principales" en ocaciones una quinta capa, que no se menciona en muchos artículos, tal vez por ser la más lejana a la Tierra. \\ \\ 
% Aquí explicaré en qué consiste cada capa 


\begin{center}
\includegraphics[width=9cm]{Atm3.jpg}
\end{center}
\vspace{1cm}

\begin{itemize}
\item Tropósfera \\
Es la capa más cercana a la superficie terrestre, donde ocurre la mayoría de fenómenos meteorológicos. Se encuentra en los primeros diez kilómetros sobre el nivel del mar, por lo que varía dependiendo del lugar en la tierra. \\ En la tropósfera se encuentra el 75\% de todo el aire, y más del 99\% del agua atmosférica; esta última disminuye con la altura. La temperatura tambien disminuye al subir, pero en cierto momento vuelve a aumentar. \\

\item Estratósfera \\
Como su nombre lo indica, la estratósfera está formada por capas o estratos, prácticamente horizontales. Su extención es entre los 9 o 17 km hasta los 50 km. Es la segunda capa de la atmósfera. \\ A medida que la altura aumenta la temperatura también aumenta, esto se debe a que los rayos ultravioleta transforman el Oxígeno en Ozono el cuál es un proceso que involucra calor. El aire, al ionizarse, se convie en un buen conductor de la electricidad y por ende, del calor. Es por esto que a cierta altura existe una relativa abundancia de ozono, que se conoce como ozonósfera, que hace que la temperatura aumente. \\

\item Mesósfera \\ Se extiende desde los 50 a 80 km y es la tercera capa de la atmósfera. Contiene sólo el 0.1\%
de la masa total del aire. Esta capa es la más fría de la atmósfera, pudiendo alcanzar los -80°C. \\
La mesosfera es importante por los procesos de ionización y reacciones químicas, que en ella ocurren. \\ En esta capa 
es donde se observan las estrellas fugaces, que son meteoroides que se han desintegrado en la termósfera. \\
%%%%%%%%   Falta gamma
\item Termósfera \\
Se hubica entre los 80 o 90 km, hasta los 500km. \\ Desde su parte inferior, donde la temperatura oscila entre los -80°C, la temperatura comienza a subir asintóticamente hasta alcanzar los 1000°K o 2000°K a los 300km de altura aproximadamente; aunque la temperatura varia dependiendo de la actividad solar. Este aumento en la temperatura se debe a la absorción de rayos X y que convierten las moléculas del aire en radicales libres, iones y electrones. \\

\item Exósfera \\
Esta capa compone el límite superior de la atmósfera. Se encuentra de los 500km hasta límites indeterminados. \\
Esta capa representa la transición de gases hacia el espacio exterior. \\

\end{itemize}

También se puede clasificar la estructura de la atmósfera en dos grandes diviciones, homósfera y heterósfera. \\  
\begin{itemize}
\item [$*$]Homósfera \\
Es la capa inferior de la atmósfera terrestre, ocupa los primeros 100km aproximadamente. \\
Esta capa se caracteriza por matener una concentración constante de la mayoria de los gases que la componen, esto gracias a fenómenos de mezcla convectiva y turbulenta. Los únicos gases que no cumplen esta regla, son el vapor de agua y el ozono. \\

\item [$*$]Heterósfera  \\ 
La heterósfera se extiende desde los 100km hasta el límite superior de la atmósfera (10,000km aproximadamente). \\ Esta capa está estratificada en diversas capas que varían en  su composición \\ 
\begin{itemize}
\item 100-400km, capa de nitrógeno
\item 400-1,100km, capa de oxígeno
\item 1,100-3,500km, capa de helio
\item 3,500-10,000, capa de hidrógeno
\end{itemize}

\vspace{1.3cm}

\subsection*{\Large Composición de la atmósfera}
La atmósfera, como sabemos se compone de una gran variedad de gases, los cuales varian en cantidad y en nivel de daño que causa. En la siguiente imagen se muestran los porcentajes que hay de cada gas, con respecto al total de gases en la atmósfera.
\\
\begin{center}
\includegraphics[width=10cm]{Atm2.png}
\end{center}
\end{itemize}
\vspace{1.75cm}
\subsection*{¿Cómo se estudia la atmósfera?}
Para saber el comportamiento de la atmósfera, en base a sus variables físicas, se utiliza un aparato llamado globo meteorológico, el cual consiste en un globo equipado con un aparato llamado radiosonda. \\ Esta radiosonda proporciona información de presión atmosférica, humedad, temperatura y viento. \\
También se hacen predicciones por medio de cáclulos matemáticos, pero para esto se ocupa información adicional.
\vspace{1cm}
\subsection*{Conclusiónes}
En la atmósfera ocurren un sin fin de procesos y fenómenos físicos y químicos. \\
Es interesante el hecho de que haya tantas capas y que varíen tanto entre sí, es decir, no existe frontera entre ellas y, sin embargo, se mantienen no se mezclan.

\vspace{1.8cm}

\begin{thebibliography}{9}
\bibitem{m1} \textsc{http://oa.upm.es} \textit{Termodinámica de la atmósfera}, 28/Enero/2017.
\bibitem{en.wikipedia.org} \textit{Globo meteorológico}, 28/Enero/2017.
\bibitem{http://climate.ncsu.edu} \textit{Structure of the Atmosphere}, \textit{Composition of the Atmosphere}, 28/Enero/2017. 
\end{thebibliography}




\end{document}