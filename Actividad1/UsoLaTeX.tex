\documentclass[a4paper]{article}

%% Language and font encodings
\usepackage[spanish]{babel}
\usepackage[utf8]{inputenc}
\usepackage[T1]{fontenc}


%% Sets page size and margins
\usepackage[a4paper,top=3cm,bottom=2cm,left=3cm,right=3cm,marginparwidth=1.75cm]{geometry}

%% Useful packages
\usepackage{amsmath}
\usepackage{graphicx}
\usepackage[colorinlistoftodos]{todonotes}
\usepackage[colorlinks=true, allcolors=blue]{hyperref}

\title{Uso de LaTeX}
\author{Paul D. Cañez Miranda}

\begin{document}
\maketitle

%%%%%%%%%%%%%%%%%%Falta contestar preguntas

\subsection*{¿Cuál es tu primera impresión del uso de LaTeX?}
Es un editor muy completo, se pueden hacer una infinidad de cosas. 


\subsection*{¿Qué aspectos te gustaron más?}
Me gustó que lo que se escribe es el código. Así se puede saber mejor lo que se está haciendo (con más presición).

\subsection*{¿Qué no pudiste hacer en LaTeX?}
No pude hacer muchas cosas, el hecho de que se puedan hacer muchas cosas, implica que hay muchos comandos. Se pueden buscar en internet, pero se ocupa tiempo para conocerlos. \\ No pude cambiar la fecha a español (en mi trabajo sobre la atmósfera).

\subsection*{¿Cómo comparas a LaTeX con otros editores?}
Es mejor en cuanto a cantidad y calidad de herramientas y funciones. Pero ocupa más tiempo familiarisarte con él. 

\subsection*{¿Qué es lo que más te llamó la atención en el desarrollo de esta actividad?}
El poder aprender a usar LaTeX. 

\subsection*{¿Qué cambiarías en esta actividad?}
No creo que deba cambiarse nada. 
\subsection*{¿Qué consideras que falta en esta actividad?}

\subsection*{¿Puedes compartir alguna referencia que consideres útil y no se haya contemplado?}
Pienso que las referencias proporcionadas fueron las precisas.



%%%%%Comentario adicional
Es un buen trabajo para comenzar con LeTeX.


\end{document}